\documentclass[a4, 10pt]{article}
\usepackage{csquotes}
%opening
\title{\textbf{MCSE 6309 And EMoS 6309:\\ Machine Learning }}

\date{June 2019}

\begin{document}

\maketitle

\section*{Course Information}

\begin{enumerate}
	\item \textbf{Course Instructor}: Dina Machuve(PhD)
	\item \textbf{Course Assistant}: Neema Mduma
\end{enumerate}

\section{Course Description}
The goal of this course is to provide an overview of the state-of-art algorithms used in
machine learning. In recent years, many successful applications of machine learning
have been developed, ranging from data-mining programs that learn to detect
fraudulent credit card transactions, to autonomous vehicles that learn to drive on
public highways, and computer vision programs that can recognize thousands of
different object types. At the same time, there have been important advances in the
theory and algorithms that form the foundation of this field. This course is divided into
three parts, a)the Mathematical Foundations for Machine Learning b) Machine Learning, and c) Practicals. This course brings the mathematical foundations of basic machine learning concepts
will provide the mathematical background, applied to four central machine learning problems, to make it easier to read other machine learning textbooks. 

\section{Course Outline}
\textbf{Part 1: Mathematical Foundations}
\begin{enumerate}
	\item Linear Algebra
	\item Analytic Geometry
	\item Vector Calculus 
	\item Statistics and Probability Theory 
	\item Optimization 
\end{enumerate}
\textbf{Part 2: Machine Learning}
\begin{enumerate}
	\item Linear Regression
	\item Classification with Logistic Regression
	\item Graphical Models
	\item Dimensionality Reduction with Principal Component Analysis
	\item Model Selection
	\item Gaussian Process Regression
	\item Bayesian Optimization
	\item Sampling
\end{enumerate}
\textbf{Part 3: Practicals}
\begin{enumerate}
	\item Python, Pandas and Jupyter Tutorial
	\item Statistics of datasets
	\item Linear regression
	\item Logistic regression
	\item Principal component analysis
\end{enumerate}		
\section{Prerequisites}
Basic knowledge of Python programming language is required. Basic knowledge of
probability/statistics, calculus and linear algebra is required.

\section*{References}
\begin{enumerate}
	\item Marc Deisenroth, A. Aldo Faisal, Cheng Soon Ong, \enquote{Mathematics for Machine Learning}, \textit{https://mml-book.com/}, To be published by Cambridge University Press.
	\item Simon Rogers, Mark Girolami, \enquote{A First Course in Machine Learning}, 2nd Edition, CRC 2016
	\item Christopher M. Bishop, \enquote{Pattern Recognition and Machine Learning},
	Springer, 2006
	\item Paul Barry, \enquote{Head First Python}, O'Reilly, 2010

\end{enumerate}

\section{Academic Integrity}
The NM-AIST upholds the standards of honesty and integrity from all members of the
academic community. The policy covers:
\begin{itemize}
	\item Plagiarism: intentionally or unintentionally representing the words or ideas of
	another person as your own; failure to properly cite references; manufacturing
	references.
	\item Submitting a paper written (entirely or even a small part) by another person or
	obtained from the Internet.
	\item Plagiarism is plagiarism: it does not matter if the source being copied is on the
	Internet, from a book or textbook, or from quizzes or problem sets written up
	by other students.
	\item The penalties for violation of the policy may include a failing grade on the
	assignment/project and/or a failing grade in the course.
\end{itemize}

\section{Homework Assignments and Term Project}
There will be four homework assignments, one class test, an open-ended course
project, and final university examination.
All homework assignments are \textbf{INDIVIDUAL}! You may discuss the problems with
your classmates, but you must submit individual homework assignments. Please
acknowledge all sources you use in the assignments (papers, code or ideas from
someone else).
\begin{itemize}
	\item Each homework assignment will include written and programing portions.
	\item Both written and programming assignments will be submitted electronically.
	\item Programming assignments will be in Python.
	\item In the term project, each group of two students will investigate some interesting
	aspect of machine learning or apply machine learning to a problem that interests you.
		
\end{itemize}

\section{Grading breakdown}
The course grades will be broken down as follows: Four homework assignments with written and programming components: 20\%; Midterm examination: 10\%; Final project: 10\%; and Final university examination: 60\%.

\end{document}
